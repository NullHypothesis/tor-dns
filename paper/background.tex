\section{Background}
\label{sec:background}

\subsection{How Tor handles DNS}
To preserve their anonymity and prevent DNS leakage, Tor clients must send their
DNS requests also over Tor.  Since Tor does not transport UDP packets, there is
a workaround to ensure that DNS requests go over Tor.

First, applications such as Tor Browser establish a connection to the SOCKS
proxy exposed by the local Tor client.  Using the SOCKS protocol, applications
instruct the Tor client to establish a circuit to a given domain and
port.\footnote{SOCKS in version 4a and 5 supports connection initiations using
domain names in addition to IP addresses.} The Tor client then selects an exit
relay whose exit policy matches the given domain and port.  Next, the client
sends a \texttt{BEGIN\_DIR} Tor cell to the exit relay, instructing it to
resolve the domain name and establish a TCP connection to the given port.  After
successfully establishing a connection, the exit relay responds with a
\texttt{RELAY\_CONNECTED} cell to the local Tor client.  From then on, data can
be exchanged with the intended destination.

As of Dec 2015, exit relays resolve domain asynchronously and both the exit
relay and the client maintain a caching layer around the resolution code to
speed up repeated lookups.  Exit relays send their DNS requests to the system
resolver, which is in \texttt{/etc/resolv.conf} on Linux systems.  The system
resolver is not set by Tor, and contains whatever the exit relay operator
configured, e.g., the ISP's resolver, or well-known, public resolvers such as
8.8.8.8.
