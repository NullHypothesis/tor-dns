\section{Introduction}
\label{sec:introduction}

% High-level motivation.
We have yet to learn how to build anonymity networks that are resistant to
global adversaries while also providing low latency.  Remailer systems like
Mixmaster and Mixminion gave up low latency in favor of strong anonymity.
Considering latency crucial for adoption, the Tor network treaded the opposite
path, trading off strong anonymity for low latency.  This design decision
enables latency-sensitive applications such as web browsing, but at the cost of
not being able to protect against adversaries that can observe traffic both
entering and exiting the anonymity network, which facilitates deanonymization.
While such adversaries are not in the threat models of low-latency networks,
that does not mean that they are not a concern.  Indeed, semi-global adversaries
are a reality~\cite{Farrell2014a}, and even smaller adversaries are a
concern~\cite{Johnson2013a}.  A sound understanding of semi-global adversaries
is necessary to better protect Tor users and mitigate attacks.

% More specific problem.
Academia has spent considerable effort on quantifying the threat of correlation
attacks~\cite{Johnson2013a,Murdoch2007a}.  Past work, however, only focused on
the TCP flows between clients and servers such as HTTP requests, BitTorrent
connections, or IRC sessions.  The DNS requests that typically precede TCP
connections received no attention.  This is a critical oversight because the
nature of DNS differs in fundamental ways from subsequent TCP flows.  Depending
on the setup of a Tor exit relay, DNS requests can traverse parts of the
Internet unknown to the subsequent TCP connection.  This aids an attacker that
can observe occasional DNS requests but no exit traffic to link both ends of the
communication.  Figure~\ref{fig:overview} illustrates an example in which an
adversary monitors the connection between a user and her guard relay, and the
exit relay and its DNS resolvers or servers.  This to date uncharted territory
is the topic of this work.

\begin{figure}[t]
	\centering
	\includegraphics[width=0.65\linewidth]{figures/attack-concept.pdf}
	\caption{Past traffic correlation studies have focused on linking the TCP
		stream entering the Tor network to the one(s) exiting the network.  We
		show that an adversary can also link the associated DNS traffic, which
		can be exposed to many more ASs than the TCP stream.}
	\label{fig:overview}
\end{figure}

% Summary of our paper.
We first set out to explore DNS resolution on Tor exit relays.  By developing a
new method to identify all exit relays' DNS resolver, we learn that Google
currently gets to see almost 40\% of all DNS requests exiting the Tor network.
Next, we are interested in the exposure of DNS requests.  Who can observe the
DNS requests of Tor exit relays?  To answer this question, we simulate DNS
resolution for the Alexa Top 1,000 domains, which reveals that the DNS
resolution process for half of these domains traverses numerous autonomous
systems that are not traversed for the subsequent HTTP connection to the web
site.  Armed with a better understanding of DNS's role in Tor, we show how to
augment existing website fingerprinting (WF) attacks with observed DNS requests,
yielding perfectly precise WF+DNS attacks for unpopular websites.
Finally we investigate the impact of our WF+DNS attacks at Internet-scale
with the help of the TorPS simulator~\cite{TorPS}.
By employing a novel method to run traceroutes from
exit relays, we believe our results to be significantly more accurate and
comprehensive than previous work.

% Comparison to past work.
We improve on the state of the art by showing that DNS requests pose a threat
that future work on correlation attacks should consider.  Our measurements
provide us with a better understanding of the Tor network that can aide in
future design decisions.  Finally, we propose a new measurement method to
evaluate the impact of traffic correlation attack, significantly improving on
past work.  Our work \first serves as guidance to Tor exit relay operators,
\second improves state-of-the-art measurement techniques, \third foo, \fourth
and sheds light on previously uncharted territory of the Tor network.  To
foster future work and facilitate the replication of our results, we publish
both our code and datasets.\footnote{We redacted the link to our project page to
anonymize our paper submission.} In summary, we make the following
contributions:
\begin{itemize}
	\item We develop a novel method to identify the DNS resolver of exit relays,
		revealing that 40\% of DNS request exiting the Tor network go to
		Google's public DNS resolvers.

	\item We quantify the exposure of the DNS resolution process to
		network-level adversaries.  Our results show that for numerous popular
		websites, 70\% of all traversed ASs are only visited for DNS requests.

	\item We show how existing WF attacks can be combined with observed DNS
	requests by a network-level adversary to yield perfectly precise
	WF+DNS attacks for unpopular websites.

	\item We develop a new measurement method to evaluate the threat of
		network-level adversaries. We employ the RIPE Atlas~\cite{atlas}
		platform to achieve previously unprecedented path coverage and accuracy.
\end{itemize}

The rest of this paper is organized as follows.  We begin by discussing related
work in Section~\ref{sec:related_work}, and providing background in
Section~\ref{sec:background}.  We then shed light on the landscape of DNS in Tor
in Section~\ref{sec:landscape}.  Section~\ref{sec:attack} discusses our WF+DNS
attacks, which we evaluate in Section~\ref{sec:analysis}.  We
then model the Internet-scale impact of our attacks in
Section~\ref{sec:internet-scale}.  Finally, we discuss our work in
Section~\ref{sec:discussion} and conclude the paper in
Section~\ref{sec:conclusion}.
