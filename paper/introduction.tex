\section{Introduction}
\label{sec:introduction}

% High-level motivation.
We have yet to learn how to build anonymity networks that both
resist global adversaries and provide low latency.  Remailer systems
such as Mixmaster~\cite{mixmaster} and
Mixminion~\cite{Danezis2003a} eschew low latency in favor of
strong anonymity.
In contrast, Tor~\cite{dingledine2004a} trades off strong anonymity to
achieve low latency; Tor therefore
enables latency-sensitive applications such as web browsing but is
vulnerable to
adversaries that can observe traffic both
entering and exiting its network, thus enabling deanonymization.
Although Tor does not consider global adversaries in its threat model,
adversaries that can observe traffic for extended periods of time in
multiple network locations (\ie, ``semi-global'' adversaries) are a real
concern~\cite{Farrell2014a,Johnson2013a}; we need to better understand
the nature to which these adversaries exist in operational networks and
their ability to deanonymize users.

% More specific problem.
Past work has quantified the extent to which an adversary that
observes TCP flows between clients and servers (\eg, HTTP requests,
BitTorrent connections, and IRC sessions) can correlate traffic flows
between the client and the entry to the anonymity network and between
the exit of the anonymity network and its ultimate
destination~\cite{Johnson2013a,Murdoch2007a}. The ability to correlate
these two flows---a so-called {\em correlation attack}---can link the
sender and receiver of a traffic flow, thus compromising the anonymity
of both endpoints. Although TCP connections are an important part
of communications, the Domain Name System (DNS) traffic is also
quite revealing: for example, even loading a single webpage can generate
hundreds of DNS requests to many different domains. No previous analysis
of correlation attacks has studied how DNS traffic can exacerbate
these attacks.

DNS traffic is highly relevant for correlation attacks because it
often traverses completely different paths
and autonomous systems (ASes) than the subsequent corresponding TCP
connections.  An attacker that can observe occasional DNS
requests may still be able to link both ends of the communication, even
if the attacker cannot observe TCP traffic between the exit of the
anonymity network and the server.
Figure~\ref{fig:overview} illustrates how an adversary may
monitor the connection between a user and the guard relay, and between the exit
relay and its DNS resolvers or servers.  This
territory---to-date, completely unexplored---is the focus of this work.

\begin{figure}[t]
	\centering
	\includegraphics[width=0.65\linewidth]{figures/attack-concept.pdf}
	\caption{Past traffic correlation studies have focused on linking the TCP
		stream entering the Tor network to the one(s) exiting the network.  We
		show that an adversary can also link the associated DNS traffic, which
		can be exposed to many more ASes than the TCP stream.}
	\label{fig:overview}
\end{figure}

% Summary of our paper.
We first explore how Tor exit relays resolve DNS names.  By developing a new
method to identify all exit relays' DNS resolvers, we learn that Google
currently sees almost 40\% of all DNS requests exiting the Tor network.  Second,
we investigate which organizations can observe DNS requests that originate from
Tor exit relays.  To answer this question, we emulate DNS resolution for the
Alexa Top 1,000 domains from an autonomous system that is popular among exit
relays.  We find that DNS resolution for half of these domains traverses numerous
ASes that are not traversed for the subsequent HTTP connection to the web site.
We further introduce a new method to perform traceroutes from the networks where
exit relays are located, making our results significantly more accurate and
comprehensive than previous work.  Next, we show how the ability to observe DNS
traffic from Tor exit relays can augment existing website fingerprinting
attacks, yielding perfectly precise \name attacks for unpopular websites.
Finally, we use the Tor Path Simulator (TorPS)~\cite{TorPS} to investigate the
effects of Internet-scale \name attacks.

% Comparison to past work.
We demonstrate that DNS requests significantly increase the opportunity
for adversaries to perform correlation attacks. This finding should
encourage future work on correlation attacks to consider both TCP
traffic and the corresponding DNS traffic; future design decisions
should also be cognizant of this threat.  The measurement methods we use
to evaluate the effects of traffic correlation attacks are also more
accurate than past work. Our work \first serves as guidance to Tor exit
relay operators and Tor network developers, \second improves
state-of-the-art measurement techniques for analysis of correlation attacks, and \third
provides even stronger justification for introducing website fingerprinting defenses in
Tor.  To foster future work and facilitate the replication of
our results, we publish both our code and datasets.\footnote{Our project page is
available at \url{https://nymity.ch/dns-traffic-correlation/}.}
In summary, we make the following contributions:
\begin{itemize}

\item We show how existing website fingerprinting attacks can be
  augmented with observed DNS requests by an AS-level adversary to
  yield perfectly precise \name attacks for unpopular websites.

\item We develop a method to identify the DNS resolver of exit
  relays. We find that Tor exit relays comprising 40\% of Tor's exit
  bandwidth rely on Google's public DNS servers to resolve DNS queries.

\item We quantify the extent to which DNS resolution exposes Tor users
  to additional AS-level adversaries who are not on the path between the
  sender and receiver.  We find that for the Alexa top 1,000 most
  popular websites, 60\% of the ASes that are on the paths between the
  exit relay and the DNS servers required to resolve the sites' domain
  names are not on the path between the exit relay and the website.

\item We develop a new measurement method to evaluate the extent to
  which ASes are on-path between exit relays and DNS resolvers. We use
  the RIPE Atlas~\cite{atlas} platform to achieve previously
  unprecedented path coverage and accuracy for evaluating the
  capabilities of AS-level adversaries.
\end{itemize}
\noindent
The rest of this paper is organized as follows.
Section~\ref{sec:background} presents background, and
Section~\ref{sec:related_work} relates our study to previous work.  In
Section~\ref{sec:landscape}, we shed light on the landscape of DNS in
Tor.  Section~\ref{sec:attack} discusses our \name attacks, which we
evaluate in Section~\ref{sec:analysis}.  We then model the
Internet-scale impact of our attacks in
Section~\ref{sec:internet-scale}.  Finally, we discuss our work in
Section~\ref{sec:discussion} and conclude the paper in
Section~\ref{sec:conclusion}.
