\section{Introduction}
\label{sec:introduction}

% High-level motivation.
Low-latency anonymity networks such as Tor~\cite{Dingledine2004a} trade off
strong anonymity for lower latency, making the network useful for all sorts of
web activity.  This comes at the cost of not being able to protect against
powerful adversaries that can observe both ends of the communication, allowing
them to link ingress to egress traffic, thus deanonymizing users.  While
low-latency networks do not consider such adversaries in their threat models,
we know that they are a reality.  In a classified document, an NSA analyst
wrote ``With manual analysis we can de-anonymize a very small fraction of Tor
users.''~\cite{torstinks} Understanding and mitigating the threat of such
adversaries is an important research topic.

% More specific problem.
Academia has spent considerable effort on quantifying the threat of correlation
attacks.  Past work only focused on \emph{TCP flows} between a client and its
destination such as HTTP requests, BitTorrent connections, or IRC sessions.
Preceding \emph{DNS requests}, however, have been ignored entirely.  This is a
critical oversight because the nature of DNS differs in fundamental ways from
subsequent TCP flows.  Depending on the setup of a Tor exit relay, DNS requests
can traverse different parts of the Internet than the subsequent TCP
connection, allowing an attacker that can observe occasional DNS requests but
no exit traffic to link both ends of the communication.
Figure~\ref{fig:overview} illustrates an example.

\begin{figure}[t]
	\centering
	\includegraphics[width=0.7\linewidth]{figures/attack-concept.pdf}
	\caption{Past traffic correlation studies have focused on linking the TCP
		stream entering the Tor network to the one(s) exiting the network.  We
		show that an adversary can also link the associated DNS traffic, which
		can be exposed to many more ASs than the TCP stream.}
	\label{fig:overview}
\end{figure}

% Contributions.
In this paper, we first set out to quantify the issue.  Deploying a new method
to identify the DNS resolvers that Tor exit relays use, we show that Google
currently handles 40\% of all DNS requests of the Tor network.  Around 10\% of
DNS requests are handled by resolvers on exit relays, but these resolvers send
requests that traverse large parts of the Internet.  According to our results,
the iterative DNS resolution process for half of Alexa's top 1,000 domains
traverses numerous autonomous systems that are not traversed for the subsequent
HTTP connection to the web site.  Armed with a better understanding of how Tor
exit relays use DNS, we adapt a website fingerprinting classifier that takes as
input captured traffic that is entering the Tor network, augmented with DNS
requests that are captured somewhere else on the Internet.  We evaluate our
website fingerprinting attack, and investigate the impact at Internet-scale
using the TorPS simulator in combination with a novel approach to run
traceroutes from exit relays.

% Comparison to past work.
We improve on the state of the art by showing that DNS requests pose a threat
that future work on correlation attacks should consider.  Our measurements
provide us with a better understanding of the Tor network that can aide in
future design decisions.  Finally, we propose a new measurement method to
evaluate the impact of traffic correlation attack, significantly improving on
past work.

In summary, our work makes the following contributions:
\begin{itemize}
	\item We shed light on how Tor exit relays resolve domains, revealing that
		40\% of DNS request exiting the Tor network go to Google's public DNS
		resolver.

	\item We present a novel website fingerprinting attack.  We augment our
		classifiers input data with DNS requests observed by a network-level
		adversary to reduce the open world scenario.

	\item We develop a new measurement method to evaluate the threat of
		network-level adversaries.  We employ the RIPE Atlas~\cite{atlas}
		platform to achieve previously unprecedented path coverage.

	\item We quantify the threat of DNS-based traffic correlation attacks by 
\end{itemize}

The rest of this paper is structured as follows.
Section~\ref{sec:related_work} begins by presenting related work, followed by
background in Section~\ref{sec:background}.  We seek understand the landscape
of DNS in Section~\ref{sec:landscape}.
