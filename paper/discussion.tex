\section{Discussion}
\label{sec:discussion}

\subsection{Mitigation}
We now discuss strategies to mitigate the issues raised by our work.  We
distinguish between short-term solution that could be implemented quickly
(\S~\ref{sec:short-term}), and long-term solution that need significantly more
work (\S~\ref{sec:long-term}).

\subsubsection{Short-term solutions}
\label{sec:short-term}
Exit relays seem stuck in a predicament, having to choose between running their
own resolver (which exposes DNS queries to network adversaries) or using a
third-party resolver (which exposes DNS queries to a third party).  If we were
given the choice to reconfigure the DNS setup of all exit relays, what would we
do?  Clearly, the goal is to minimize exposure of DNS requests, but there are
several dimensions to this.  Ignoring substantial DNS protocol improvements for
now, we can envision three extreme scenarios, in which \emph{all} exit relays
use either \first Google's DNS resolver; \second their own, local resolver; or
\third the resolver provided by their ISP.  If all exit relays would use Google,
the company would obtain metadata about the activity of all Tor users, a clear
violation of Tor's design goal of distributing trust.  We believe that this
scenario is clearly to avoid.\footnote{To make matters worse, Fifield et
al.'s~\cite{Fifield2015a} censorship circumvention system meek leverages
Google's cloud infrastructure to reach the Tor network.  Inevitably, many of
the thousands of meek clients will select exit relays that use Google's DNS
resolver, which means that Google gets to see both traffic entering and,
partially, exiting the Tor network.} Next, let us imagine a Tor network that
only uses local resolvers.  In that case, Tor is fully independent of
third-party resolvers, at the cost of the DNS resolution process being exposed
to the network, allowing several parties to learn what domain a Tor user is
looking up.  Finally, all exit relays could simply use their ISP-provided
resolver.  This would minimize the exposure of DNS requests as resolvers are
frequently in the same AS as exit relays, and network-level adversaries would be
unable to distinguish between DNS requests from exit relays and unrelated ISP
customers.  This setup comes at the cost of potentially broken DNS resolvers.
Besides, just a few ASs---OVH, for example---host a disproportionate amount of
exit relays, turning them into the very centralized data trough we seek to
avoid.  Table~\ref{tab:setup-comparison} compares the key aspects of all three
setups.

\begin{table}[t]
	\caption{A comparison between three hypothetical DNS setups for all exit
	relays.  Full circles are desirable.}
	\label{tab:setup-comparison}
	\centering
	\begin{tabular}{l c c c}
	\toprule
	\textbf{Setup} &
	\begin{tabular}{@{}c@{}}\textbf{Network-level}\\\textbf{protection}\end{tabular} &
	\begin{tabular}{@{}c@{}}\textbf{Level of}\\\textbf{Centralization}\end{tabular} &
	\begin{tabular}{@{}c@{}}\textbf{Response}\\\textbf{quality}\end{tabular} \\
	\midrule
	All Google & \RIGHTcircle & \Circle & \CIRCLE \\
	All Local & \Circle & \CIRCLE & \CIRCLE\\
	All ISP & \CIRCLE & \RIGHTcircle & \RIGHTcircle \\
	\bottomrule
	\end{tabular}
\end{table}

Considering the above, we believe that exit relay operators should avoid
third-party resolvers such as Google and OpenDNS.  Instead, they should either
use the resolvers provided by their ISP, or run their own---in particular, if
their ISP already hosts many other exit relays.

Besides making recommendations to exit relay operators, we can remotely
influence the cache of each exit relay's resolver.  For example, using exitmap,
we can resolve the domain sensitive.org over each exit relay continuously, right
before its TTL is about to expire.  That way, once a Tor user connects to
sensitive.org, the exit relay can serve the request from its cache instead of
exposing it to third parties.  Clearly, this approach does not scale, but it
allows us to eliminate DNS-based correlation attacks for a select number of
sites.

\begin{itemize}
	\item Running local resolver could be dangerous since some resolvers write stuff to disk:
		\url{https://github.com/NullHypothesis/exitmap/commit/fa1cd0ba6ce3389d3d6fe1dfc7144e2747320944#commitcomment-16886249}
	\item Would qname minimisation help?
	\item Ideally, use only onion services, or perhaps OnionNS, or the GNU Name
		System~\cite{Wachs2014a}.
	\item Can exit relays use private information retrieval to fetch domain
		names anonymously?
	\item Is there a trade-off between enforced min DNS TTL and the impact of
		DNS poisoning?
	\item A tool that forces Tor-exits to regularly resolve uniquely identifying
		domain names?
	\item T-DNS~\cite{Zhu2015a}
\end{itemize}

\subsubsection{Long-term solutions}
\label{sec:long-term}
Practical defenses are on the horizon.  Zhu et al.~\cite{Zhu2015a} proposed
T-DNS, which transports the DNS protocol over TLS and TCP.  Since the protocol
uses TCP, it can be sent over Tor without any modification.  The TLS layer
guarantees confidentiality between Tor clients and their resolvers.

Meanwhile, Google is experimenting with DNS over HTTPS.

Better fingerprinting defenses.

More onion services.
