\section{Related work}
\label{sec:related_work}

\paragraph{Traffic analysis methods}
Tor's threat model excludes global adversaries~\cite{Dingledine2004a}, but the
practical threat of such adversaries is an important question that academia has
spent considerable effort on answering.  In 2004, when the Tor network counted
only 33 relays, Feamster and Dingledine investigated the practical threat that
network-level adversaries pose to anonymity networks~\cite{Feamster2004a}.  In
particular, the authors considered an attacker that controls an autonomous
system that is traversed both for ingress and egress traffic, allowing the
attacker to correlate both streams.  Using AS path prediction~\cite{Gao2001a},
Feamster and Dingledine found that powerful tier-1 ISPs reduce location
diversity of anonymity networks.

In 2007, Murdoch and Zieli\'{n}ski drew attention to IXP-level adversaries, a
class of adversaries that was missing in Feamster and Dingledine's
work~\cite{Murdoch2007a}.  Murdoch and Zieli\'{n}skishowed showed that IXP
adversaries are able to correlate traffic streams, even in the presence of
packet sampling rates as low as one in 2,000.

In 2013, Johnson et al.~\cite{Johnson2013a} presented the first large-scale
study on the risk of Tor users facing relay-level and network-level
adversaries.  The authors developed a Tor path simulator (TorPS~\cite{TorPS})
that simulates Tor circuits for a number of user models the authors developed.
By combining TorPS with AS path prediction, Johnson et al. could answer
questions such as the average time until a Tor user's circuit is linked
together by an AS or IXP.

In 2015, Juen et al.~\cite{Juen2015a} questioned the accuracy of path
prediction algorithms that prior work~\cite{Johnson2013a,Feamster2004a} used to
estimate the threat of correlation attacks.  The authors compared AS path
predictions to millions of traceroutes they initiated from 25\% of Tor relays
by bandwidth at the AS level.  Only 20\% of predicted paths matched the paths
observed in traceroute, calling into question the results of prior work.  A
limitation of Juen et al.'s work is that they could not consider the reverse
path in traceroutes.  This shortcoming was addressed in 2015 by Sun et
al.~\cite{Sun2015a}.  While past work treated routing as static, Sun et al.
leveraged the dynamic nature of routing to show that network adversaries are a
bigger threat than thought.

Most recently in 2016, Nithyanand et al.~\cite{Nithyanand2016a} used AS path
prediction to evaluate the practical threat faced by users in the top 10
countries using Tor.

We improve on previous work in two significant ways; (\emph{i}) we are the
first to consider the DNS protocol for traffic analysis and evaluate its
practical threat, and (\emph{ii}) we propose a method to scale the measurement
method proposed by Juen et al.~\cite{Juen2015a}.  Our method leverages the
volunteer-run RIPE Atlas measurement platform~\cite{atlas} instead of
convincing relay operators to run third-party scripts.  This allows us to fully
automate our method and achieve previously unprecedented scale.

\paragraph{Website fingerprinting}
\cite{Juarez2014a}
Showed that many variables are ignored that have large impact on classification
and running WF system is expensive.

Most recently, in 2016, Panchenko et al. show that web\emph{page}
fingerprinting lacks precision in the open world while web\emph{site}
fingerprinting remains practical~\cite{Panchenko2016a}.
