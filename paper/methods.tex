\section{Methods}
\label{sec:methods}

\subsection{Threat model}
We consider the following adversaries.
\begin{description}
	\item[Network-level adversary] An adversary that monitors at least
		one autonomous system, e.g., ISP, VPS provider, government.
	\item[Relay-level adversary] An adversary that runs at least one Tor relay.
	\item[DNS provider] An adversary that operates the DNS resolver used
		by exit relays, e.g., Google.
	\item[Active] To which extent to we consider active, content-modifying
		adversaries?  Attackers that see DNS can poison response and redirect
		victim to attacker-controlled domain.  That turns problem into trivial,
		classical end-to-end correlation.  Still, probably reason enough to tell
		exit relays that they should use DNSSEC.  Also, we might spot these
		attacks using exitmap.
\end{description}

We do not consider the case where Tor users misconfigure their client and leak
DNS requests unintentionally.

\subsection{Exposure at the Guard side}
\begin{itemize}
	\item Bad guards
	\item recognise dns requests by traffic analysis on the wire.  probably
		difficult because of optimistic data?  can we do more than just timing?
	\item can flow watermarking help?  see houmansadr's
		work~\cite{Houmansadr2011a}
\end{itemize}

\subsection{Internet map}
\begin{itemize}
	\item Where do we get our AS and IXP graph from?
	\item Johnson et al.~\cite[\S 5.2]{Johnson2013a} used RouteViews and the CAIDA IPv4
		Routed /24 AS Links Dataset, resulting in 44,605 ASs connected by
		305,381 links.
\end{itemize}

\subsection{Traceroute dataset}
\label{sec:traceroute-dataset}
\begin{itemize}
	\item Find machines that are topologically close to DNS resolvers
		used by exit relays.
	\begin{itemize}
		\item RIPE Atlas probes.
		\item Virtual private systems.
		\item PlanetLab nodes.
		\item Ask exit operators to run traceroutes for us.
	\end{itemize}
	\item Run traceroutes to DNS servers to determine path coverage.
	\item Determine ``AS inflation factor.''
\end{itemize}

\subsection{DNS packet sizes at entry guard}
\begin{itemize}
	\item Send many DNS requests over entry guard.
	\item Capture them on the wire and look at them.
	\item What's the best way to filter out the noise?
\end{itemize}

\subsection{Practical attacks}
We leverage the following building blocks:
\begin{itemize}
	\item Traffic analysis done on an entry guard.  Can DNS requests be isolated
		reliably?
	\item Access to DNS root, .com, and .net.
	\item Ability to use DNS resolver's cache as an oracle.
	\item Ability to enumerate what resolver's an exit relay uses.
	\item Resolvers using EDNS.
	\item Users in country X are likely to resolve domains of country X.
\end{itemize}
